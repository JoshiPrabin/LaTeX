\begin{center}
\section{LITERARY REVIEW}
\end{center}
In the field of academia, efficiency is one of the core mantra. To make learning more and more efficient, we have witnessed a surge in the development of online platforms to facilitate just that. In this very field, we aim to provide a simple, smart and secure solution.
\vspace{0.2in}

\subsection{Existing}
\textbf{Canvas by Instructure:}\\
 "Canvas" by Instructure is a widely adopted Learning Management System (LMS) used by educational institutions globally. Canvas streamlines resource distribution, course management, and student interaction, much like the SAP's objectives. Canvas has demonstrated the value of providing students and educators with a unified space to access course materials, interact with peers, submit assignments, and receive feedback, all contributing to enhanced student engagement and academic outcomes.\cite{canvas}\\
 \\
 \textbf{Microsoft Teams:}\\
 Microsoft Teams, which, while originally designed for team collaboration in professional settings, has gained traction in the educational sector. With features like virtual rooms, document sharing, and real-time communication, Teams mirrors the collaborative aspects of the SAP proposal. The platform's versatility showcases the effectiveness of a comprehensive tool in fostering interaction and resource sharing among users, aligning with the SAP's vision of fostering student collaboration.\cite{microsoft}
 \newpage

 \subsection{Proposed}
 The proposed SAP aims to build upon the strengths of existing platforms while addressing specific challenges unique to the student academic experience. By integrating resource management, virtual room collaboration, personalized dashboards, and timely notifications, the SAP offers a holistic approach to student engagement and academic success. This is a step towards bridging the gap between classroom and remote learning, promoting technology literacy, and empowering both students and instructors to optimize their educational journey.\\
 The integration of resource management within the SAP ensures that students have a consolidated platform to access course materials, assignments, and reference materials, eliminating the frustration of navigating multiple systems. The inclusion of virtual room collaboration goes beyond mere content distribution, creating interactive spaces where students can discuss topics, collaborate on projects, and learn from one another, fostering a sense of community and active participation.

 \subsubsection{Functional Requirements:}
    \begin{itemize}
        \item The system must allow students, instructors, and administrators to log in securely with their unique credentials.
        \item The system should support password reset functionality.
        \item Students and instructors should be able to join rooms based on their courses or interests.
        \item Instructors must have the ability to create, modify, and manage rooms.
        \item Each room should display relevant information, such as room name, capacity, and resources available.
        \item Students in a room should be able to access resources such as files, links, and educational materials relevant to that room.
        \item Instructors should be able to upload and manage resources for their respective rooms.
        \item Administrators should be notified of access requests and have the ability to approve or deny them.
        \item The system should send timely notifications to users about room updates, resource availability, and access request status.
    \end{itemize}

    \subsubsection{Non-functional Requirements:}
    \begin{itemize}
        \item User data (including login credentials) must be securely stored and transmitted.
        \item Access to resources should be restricted based on room membership or authorization.
        \item The system should be responsive and capable of handling a large number of concurrent users without significant performance degradation.
        \item The user interface should be intuitive, easy to navigate, and visually appealing.
        \item The system should be accessible and usable on both desktop and mobile devices.
        \item The system should be highly available, with minimal downtime for maintenance.
        \item Data integrity must be maintained, and data loss should be prevented.
    \end{itemize}
 
\newpage